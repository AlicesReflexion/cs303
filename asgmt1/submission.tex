\documentclass[12pt]{article}
\usepackage{mathtools}
\title{CS303 Data Structures Assignment \#1}
\begin{document}

\maketitle

\section*{1.}
{\small
Choose an arbitrary \(n_0\), eg. \(n_0=1\).

\(Cn_0^3>n_0^3-5n_0^2+20n_0-10\)

\(C*1^3>1^3-5*1^2+20*1-10\)

\(C>1-5+20-10\)

\(C=6, n_0=1\)\\
}
For all \(n\), where \(n>1\), \(6n^3>n^3-5n^2+20n-10\).\\

\section*{2.}
See attached \texttt{comparegrowth.cpp}. Output:\\
\texttt{
y1: 10\\
y2: 2\\
y1: 1010\\
y2: 502\\
y1: 2010\\
y2: 2002\\
y1: 3010\\
y2: 4502\\
y1: 4010\\
y2: 8002\\
y1: 5010\\
y2: 12502\\
y1: 6010\\
y2: 18002\\
y1: 7010\\
y2: 24502\\
y1: 8010\\
y2: 32002\\
y1: 9010\\
y2: 40502\\
y1: 10010\\
y2: 50002\\
}

The results here are to be expected. y1 is initially larger because of the constant 100 rather than 5, but y2 quickly overtakes it because of how much faster \(n^2\) grows than \(n\). This would be true regardless of what constant y1 used. If \(y1=10000n +20\), y2 would still outgrow it.\\

\section*{3.}
\subsection*{3.1}
The inner loop is run \(i^2\) times, with i being every number from 0 to n-1. Therefore, we get the sum:
\(T(n)=1^2+2^2+3^2\dots+n^2\) or\\
\(T(n)=\displaystyle\sum_{i=1}^n i^2\), which is simply a geometric series, so \(T(n)=\frac{1}{6}n(n+1)(2n+1)\).\\

\end{document}
