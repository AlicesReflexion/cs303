\documentclass[12pt]{article}
\usepackage[left=2cm, right=2cm, top=1cm, bottom=2cm]{geometry}
\title{CS303 Data Structures Assignment \#3\\
\large attachments and source available at https://github.com/alexskc/cs303}
\author{Aleksander Charatonik}
\begin{document}

\maketitle

\section{}
A \texttt{const\_iterator} is useful in preventing modifying the referenced value. It's simply about const correctness, and informs what the programmer should be able to do. \texttt{iterator}, by contrast, has read-write access, and is useful in scenarios where that is necessary.

\section{}
\subsection*{a}
An \texttt{iterator}. Regardless of whether you have an array-based structure, or a linked-list structure, you need to be able to change either the value of the next item, or the pointer to the next item.
\subsection*{b}
\texttt{iterator} as well. You are modifying data, so you cannot be read-only.
\subsection*{c}
For this one, a \texttt{cost\_iterator} will suffice. We are not changing any data.
\subsection*{d}
\texttt{iterator} as well. We can avoid changing the element pointed to if we're using a linked-list structure, but we still need to change the element before it to point to the new item. And of course, in an array-based structure, we're going to be moving elements around in the array to make space for the new element.

\section{}
See attached \texttt{reverser.cpp}

\section{}
\begin{tabular}{| l | l | l |}
  \hline
  \textbf{Expression} & \textbf{Action} & \textbf{Stack} \\ \hline
  \begin{tabular}{c c c c c c c c c c c}
    10 & 2 & * & 5 & / & 6 & 2 & 5 & * & + & -\\
    \(\uparrow\)
  \end{tabular} & Push 10 & 
  \begin{tabular}{|c|}
    \hline
    10\\ \hline
  \end{tabular} \\ \hline
    \begin{tabular}{c c c c c c c c c c c}
    10 & 2 & * & 5 & / & 6 & 2 & 5 & * & + & -\\
       & \(\uparrow\)\\
  \end{tabular} & Push 2 & 
  \begin{tabular}{|c|}
    \hline
    2\\ \hline
    10\\ \hline
  \end{tabular} \\ \hline
  \begin{tabular}{c c c c c c c c c c c}
    10 & 2 & * & 5 & / & 6 & 2 & 5 & * & + & -\\
       &   & \(\uparrow\)\\
  \end{tabular} & Eval * & 
  \begin{tabular}{|c|}
    \hline
    20\\ \hline
  \end{tabular} \\ \hline
  \begin{tabular}{c c c c c c c c c c c}
    10 & 2 & * & 5 & / & 6 & 2 & 5 & * & + & -\\
       &   &   & \(\uparrow\)\\
  \end{tabular} & Push 5 & 
  \begin{tabular}{|c|}
    \hline
    5\\ \hline
    20\\ \hline
  \end{tabular} \\ \hline
  \begin{tabular}{c c c c c c c c c c c}
    10 & 2 & * & 5 & / & 6 & 2 & 5 & * & + & -\\
       &   &   &   & \(\uparrow\)\\
  \end{tabular} & Eval / & 
  \begin{tabular}{|c|}
    \hline
    4\\ \hline
  \end{tabular} \\ \hline
  \begin{tabular}{c c c c c c c c c c c}
    10 & 2 & * & 5 & / & 6 & 2 & 5 & * & + & -\\
       &   &   &   &   & \(\uparrow\)\\
  \end{tabular} & Push 6 & 
  \begin{tabular}{|c|}
    \hline
    6\\ \hline
    4\\ \hline
  \end{tabular} \\ \hline
  \begin{tabular}{c c c c c c c c c c c}
    10 & 2 & * & 5 & / & 6 & 2 & 5 & * & + & -\\
       &   &   &   &   &   & \(\uparrow\)\\
  \end{tabular} & Push 2 & 
  \begin{tabular}{|c|}
    \hline
    2\\ \hline
    6\\ \hline
    4\\ \hline
  \end{tabular} \\ \hline
  \begin{tabular}{c c c c c c c c c c c}
    10 & 2 & * & 5 & / & 6 & 2 & 5 & * & + & -\\
       &   &   &   &   &   &   & \(\uparrow\)\\
  \end{tabular} & Push 5 & 
  \begin{tabular}{|c|}
    \hline
    5\\ \hline
    2\\ \hline
    6\\ \hline
    4\\ \hline
  \end{tabular} \\ \hline
  \begin{tabular}{c c c c c c c c c c c}
    10 & 2 & * & 5 & / & 6 & 2 & 5 & * & + & -\\
       &   &   &   &   &   &   &   & \(\uparrow\)\\
  \end{tabular} & Eval * & 
  \begin{tabular}{|c|}
    \hline
    10\\ \hline
    6\\ \hline
    4\\ \hline
  \end{tabular} \\ \hline
  \begin{tabular}{c c c c c c c c c c c}
    10 & 2 & * & 5 & / & 6 & 2 & 5 & * & + & -\\
       &   &   &   &   &   &   &   &   & \(\uparrow\)\\ 
  \end{tabular} & Eval + & 
  \begin{tabular}{|c|}
    \hline
    16\\ \hline
    4\\ \hline
  \end{tabular} \\ \hline
  \begin{tabular}{c c c c c c c c c c c}
    10 & 2 & * & 5 & / & 6 & 2 & 5 & * & + & -\\
       &   &   &   &   &   &   &   &   &   & \(\uparrow\)\\
  \end{tabular} & Eval - & 
  \begin{tabular}{|c|}
    \hline
    -12\\ \hline
  \end{tabular} \\ \hline



\end{tabular}

\end{document}
