\documentclass[12pt]{article}
\usepackage[left=2cm, right=2cm, top=1cm, bottom=2cm]{geometry}
\usepackage{forest}
\title{CS303 Data Structures Assignment \#6\\
\large attachments and source available at https://github.com/alexskc/cs303}
\author{Aleksander Charatonik}
\begin{document}

\maketitle

\section{}
\begin{forest}
  [76
    [37
      [26
        [20]
        [6]
      ]
      [32
        [18]
        [28]
      ]
    ]
    [74
      [39
        [29]
      ]
      [66]
    ]
  ]
\end{forest}

\section{}
Operator \(>\) would create a min queue rather than a max queue.
\section{}
\begin{verbatim}
struct person
{
  std::string name;
  std::vector<person *> dependants;

  bool operator() (const person& left, const person& right) {
    return left.dependants.size() < right.dependants.size();
  }
};
\end{verbatim}
\section{}
\begin{forest}
  [120
    [* 50]
    [70
      [+ 30]
      [40
        [15
          [\% 5]
          [/ 10]
        ]
        [- 25]
      ]
    ]
  ]
\end{forest}
\section{}
If every symbol appeared equally frequantly, they'd all be on the bottom of the tree. Each letter would be encoded with the same number of bits, and we'd be iterating through a series of binary numbers, which would be something like ASCII.
\section{}
We can modify the existing main.cpp to do what we want. See the attached threeblindmice.cpp
\end{document}
